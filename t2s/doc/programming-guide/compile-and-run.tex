\chapter{How to compile and run}
\label{cha:compile-and-run}

Suppose a specification file, named \texttt{a.cpp}, is ready, and it contains such statements:
    \begin{markdown}
        Func SomeKernel(Place::Device);
        definition of SomeKernel
        SomeKernel.compile_to_host("interface", parameters of SomeKernel, IntelFPGA);
    \end{markdown}

And suppose a host file \texttt{host.cpp} is ready to call \texttt{SomeKernel}.

\begin{enumerate}
\item {\bf Set up environment}.
    
    Under the T2SP root directory:
    \begin{markdown}
    source setenv.sh (local | devcloud) fpga
    \end{markdown}
    
    \item {\bf Compile the specification}.
    \begin{markdown}
    # If you intend to emulate your design on a CPU:
    g++ a.cpp $COMMON_OPTIONS_COMPILING_SPEC $EMULATOR_LIBHALIDE_TO_LINK
    
    # Or if you intend to execute it on real FPGA hardware:
    g++ a.cpp $COMMON_OPTIONS_COMPILING_SPEC $HW_LIBHALIDE_TO_LINK
    \end{markdown}

    \item {\bf Run the specification to generate the kernel}.
    \begin{markdown}
    # For emulation:
    env BITSTREAM=a.aocx AOC_OPTION="$COMMON_AOC_OPTION_FOR_EMULATION" ./a.out    
    
    # Or for execution:
    env BITSTREAM=a.aocx AOC_OPTION="$COMMON_AOC_OPTION_FOR_EXECUTION" ./a.out    
    \end{markdown}
    For the kernel, this command generates an OpenCL file (named \texttt{a.cl} after name of the environment variable \texttt{BITSTREAM}), invokes Intel FPGA SDK for OpenCL to synthesize the OpenCL file into a device bitstream (\texttt{a.aocx}), and generates \texttt{interface.h/cpp} for the host to invoke the kernel, as if the kernel is a common CPU function.

    You may modify the generated OpenCL file manually. After any modification, you may re-generate the bitstream in the following commands:
    \begin{markdown}
    # For emulation:
    aoc $COMMON_AOC_OPTION_FOR_EMULATION a.cl -o a.aocx    
    
    # Or for execution:
    rm -rf a.aoc* a/
    aoc $COMMON_AOC_OPTION_FOR_EXECUTION a.cl -o a.aocx    
    \end{markdown}

    \item{\bf Comple the host file}
    \begin{markdown}
    # For emulation:
    g++ host.cpp interface.cpp $COMMON_OPTIONS_COMPILING_HOST_FOR_EMULATION -o host.out
    
    # Or for execution:
    g++ host.cpp interface.cpp $COMMON_OPTIONS_COMPILING_HOST_FOR_EXECUTION -o host.out
    \end{markdown}
    When compiling the host file, the kernel will be linked to the host code through the interface.

    \item {Run the host file}
    \begin{markdown}
    # For emulation:
    env BITSTREAM=a.aocx CL_CONTEXT_EMULATOR_DEVICE_INTELFPGA=1 \
        INTEL_FPGA_OCL_PLATFORM_NAME="$EMULATOR_PLATFORM" ./host.out

    # For execution:
    # DevCloud A10PAC FPGA only: convert the signed bitstream to be unsigned first
    #   Type `y` when prompted
        source $AOCL_BOARD_PACKAGE_ROOT/linux64/libexec/sign_aocx.sh \ 
               -H openssl_manager -i a.aocx -r NULL -k NULL -o a_unsigned.aocx
        mv a_unsigned.aocx a.aocx
    # Offload and run    
    aocl program acl0 a.aocx
    env BITSTREAM=a.aocx INTEL_FPGA_OCL_PLATFORM_NAME="$HW_PLATFORM" ./host.out
    \end{markdown}
\end{enumerate}

Note: if you want to re-run the above commands, remove the previously generated bitstream and intermediate files:
    \begin{markdown}
        rm -rf a.* a/ *interface.* *.out exec_time.txt
    \end{markdown}